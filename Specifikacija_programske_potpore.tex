\chapter{Specifikacija programske potpore}
		
	\section{Funkcionalni zahtjevi}
			
			\noindent \textbf{Dionici:}
			
			\begin{packed_enum}
				
				\item Donor
				\item Zavod
				\item Crveni Križ
				\item Administrator
				\item Razvojni tim ( Bruna Matić, Bruno Galić, Jana Matić, Jelena Lončar, Nikola Borzić, Nikola Marić, Zvonko Lelas )
				\item Asistent predmeta ( Mateja Golec )
				\item profesor predmeta ( Vlado Sruk )
				
			\end{packed_enum}
			
			\noindent \textbf{Aktori i njihovi funkcionalni zahtjevi:}
			
			
			\begin{packed_enum}
				\item  \underbar{Donor (sudionik) može:}
				
				\begin{packed_enum}
					
					\item pristupiti podacima
					\begin{packed_enum}
						
						\item pristup osobnim podacima (ime, prezime, krvna grupa)
						\item  pristupiti svojoj povijesti darivanja
						\item  pristupiti informacijama o akcijama
						\item pristupiti informacijama o količini krvi u pojedinim gradovima
						\item brisanje svog korisničkog računa
				
					\end{packed_enum}

					\item vidjeti koliki mu je period čekanja do sljedećeg darivanja
					\item prijava na akcije
					\item primanje obavijesti od zavoda u slučaju manjka krvi
					\item potvrđivanje pozivnica za darivanje
					\item pristup lokacijama na karti u kojima su organizirana doniranja krvi
					
				\end{packed_enum}
				\item  \underbar{Neregistrirani korisnik (sudionik) može:}
					\begin{packed_enum}
						
						\item vidjeti kartu lokacija za doniranje
						\item vidjeti akcije u kartici
						\item registrirati se
						\begin{packed_enum}
						
						\item kao donor
						\item  kao Crveni Križ
						\item  kao zavod
				
					\end{packed_enum}
				
					\end{packed_enum}
				\item  \underbar{Crveni Križ (inicijator) može:}
				
				\begin{packed_enum}
					
					\item izdavati akcije
					\item dodjeljivati priznanja (nagrade)
					\item evidencija darivatelja
					\item izdavanje potvrda
					\item davanje pozivnica
				\end{packed_enum}

				\item \underbar{Zavod (inicijator) može:}
				\begin{packed_enum}
					
					\item vidjeti popis donora
					\item brisati korisničke račune
					\item izdavati akcije
				\end{packed_enum}
			
				\item \underbar{Baza podataka (inicijator) može:}
				\begin{packed_enum}
					
					\item pohraniti sve podatke o donorima i njihovim ovlastima
					\item pohranjuje podatke o:
						\begin{packed_enum}
						
							\item lokacijama doniranja
							\item količinama krvi
							\item akcijama
							\item odzivima na akcije
						\end{packed_enum}
				\end{packed_enum}
			
			\end{packed_enum}
			
			\eject 
			
			
				
			\subsection{Obrasci uporabe}
				
				\textbf{\textit{dio 1. revizije}}
				
				\subsubsection{Opis obrazaca uporabe}
					\textit{Funkcionalne zahtjeve razraditi u obliku obrazaca uporabe. Svaki obrazac je potrebno razraditi prema donjem predlošku. Ukoliko u nekom koraku može doći do odstupanja, potrebno je to odstupanje opisati i po mogućnosti ponuditi rješenje kojim bi se tijek obrasca vratio na osnovni tijek.}\\
					

					\noindent \underbar{\textbf{UC$<$broj obrasca$>$ -$<$ime obrasca$>$}}
					\begin{packed_item}
	
						\item \textbf{Glavni sudionik: }$<$sudionik$>$
						\item  \textbf{Cilj:} $<$cilj$>$
						\item  \textbf{Sudionici:} $<$sudionici$>$
						\item  \textbf{Preduvjet:} $<$preduvjet$>$
						\item  \textbf{Opis osnovnog tijeka:}
						
						\item[] \begin{packed_enum}
	
							\item $<$opis korak jedan$>$
							\item $<$opis korak dva$>$
							\item $<$opis korak tri$>$
							\item $<$opis korak četiri$>$
							\item $<$opis korak pet$>$
						\end{packed_enum}
						
						\item  \textbf{Opis mogućih odstupanja:}
						
						\item[] \begin{packed_item}
	
							\item[2.a] $<$opis mogućeg scenarija odstupanja u koraku 2$>$
							\item[] \begin{packed_enum}
								
								\item $<$opis rješenja mogućeg scenarija korak 1$>$
								\item $<$opis rješenja mogućeg scenarija korak 2$>$
								
							\end{packed_enum}
							\item[2.b] $<$opis mogućeg scenarija odstupanja u koraku 2$>$
							\item[3.a] $<$opis mogućeg scenarija odstupanja  u koraku 3$>$
							
						\end{packed_item}
					\end{packed_item}
				
					
				\subsubsection{Dijagrami obrazaca uporabe}
					
					\textit{Prikazati odnos aktora i obrazaca uporabe odgovarajućim UML dijagramom. Nije nužno nacrtati sve na jednom dijagramu. Modelirati po razinama apstrakcije i skupovima srodnih funkcionalnosti.}
				\eject		
				
			\subsection{Sekvencijski dijagrami}
				
				\textbf{\textit{dio 1. revizije}}\\
				
				\textit{Nacrtati sekvencijske dijagrame koji modeliraju najvažnije dijelove sustava (max. 4 dijagrama). Ukoliko postoji nedoumica oko odabira, razjasniti s asistentom. Uz svaki dijagram napisati detaljni opis dijagrama.}
				\eject
	
		\section{Ostali zahtjevi}
		
			\textbf{\textit{dio 1. revizije}}\\
		 
			 \textit{Nefunkcionalni zahtjevi i zahtjevi domene primjene dopunjuju funkcionalne zahtjeve. Oni opisuju \textbf{kako se sustav treba ponašati} i koja \textbf{ograničenja} treba poštivati (performanse, korisničko iskustvo, pouzdanost, standardi kvalitete, sigurnost...). Primjeri takvih zahtjeva u Vašem projektu mogu biti: podržani jezici korisničkog sučelja, vrijeme odziva, najveći mogući podržani broj korisnika, podržane web/mobilne platforme, razina zaštite (protokoli komunikacije, kriptiranje...)... Svaki takav zahtjev potrebno je navesti u jednoj ili dvije rečenice.}
			 
			 
			 
	